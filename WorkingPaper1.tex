\documentclass[11pt,a4paper,]{article}
\usepackage[]{mathpazo}

\usepackage{amssymb,amsmath}
\usepackage{ifxetex,ifluatex}
\usepackage{fixltx2e} % provides \textsubscript
\ifnum 0\ifxetex 1\fi\ifluatex 1\fi=0 % if pdftex
  \usepackage[T1]{fontenc}
  \usepackage[utf8]{inputenc}
\else % if luatex or xelatex
  \usepackage{unicode-math}
  \defaultfontfeatures{Ligatures=TeX,Scale=MatchLowercase}
\fi
% use upquote if available, for straight quotes in verbatim environments
\IfFileExists{upquote.sty}{\usepackage{upquote}}{}
% use microtype if available
\IfFileExists{microtype.sty}{%
\usepackage[]{microtype}
\UseMicrotypeSet[protrusion]{basicmath} % disable protrusion for tt fonts
}{}
\PassOptionsToPackage{hyphens}{url} % url is loaded by hyperref
\usepackage[unicode=true]{hyperref}
\hypersetup{
            pdftitle={Meta-learning how to forecast~time series},
            pdfkeywords={Time Series, Forecasting, Time Series Features, Random Forest,
Meta-learning, Algorithm selection problem},
            pdfborder={0 0 0},
            breaklinks=true}
\urlstyle{same}  % don't use monospace font for urls
\usepackage{geometry}
\geometry{a4paper, text={16cm,24cm}}
\usepackage[style=authoryear-comp,]{biblatex}
\addbibresource{FBF.bib}
\usepackage{longtable,booktabs}
% Fix footnotes in tables (requires footnote package)
\IfFileExists{footnote.sty}{\usepackage{footnote}\makesavenoteenv{long table}}{}
\IfFileExists{parskip.sty}{%
\usepackage{parskip}
}{% else
\setlength{\parindent}{0pt}
\setlength{\parskip}{6pt plus 2pt minus 1pt}
}
\setlength{\emergencystretch}{3em}  % prevent overfull lines
\providecommand{\tightlist}{%
  \setlength{\itemsep}{0pt}\setlength{\parskip}{0pt}}
\setcounter{secnumdepth}{5}

% set default figure placement to htbp
\makeatletter
\def\fps@figure{htbp}
\makeatother


\title{Meta-learning how to forecast~time series}

%% MONASH STUFF

%% CAPTIONS
\RequirePackage{caption}
\DeclareCaptionStyle{italic}[justification=centering]
 {labelfont={bf},textfont={it},labelsep=colon}
\captionsetup[figure]{style=italic,format=hang,singlelinecheck=true}
\captionsetup[table]{style=italic,format=hang,singlelinecheck=true}

%% FONT
\RequirePackage{bera}
\RequirePackage{mathpazo}

%% HEADERS AND FOOTERS
\RequirePackage{fancyhdr}
\pagestyle{fancy}
\rfoot{\Large\sffamily\raisebox{-0.1cm}{\textbf{\thepage}}}
\makeatletter
\lhead{\textsf{\expandafter{\@title}}}
\makeatother
\rhead{}
\cfoot{}
\setlength{\headheight}{15pt}
\renewcommand{\headrulewidth}{0.4pt}
\renewcommand{\footrulewidth}{0.4pt}
\fancypagestyle{plain}{%
\fancyhf{} % clear all header and footer fields
\fancyfoot[C]{\sffamily\thepage} % except the center
\renewcommand{\headrulewidth}{0pt}
\renewcommand{\footrulewidth}{0pt}}

%% MATHS
\RequirePackage{bm,amsmath}
\allowdisplaybreaks

%% GRAPHICS
\RequirePackage{graphicx}
\setcounter{topnumber}{2}
\setcounter{bottomnumber}{2}
\setcounter{totalnumber}{4}
\renewcommand{\topfraction}{0.85}
\renewcommand{\bottomfraction}{0.85}
\renewcommand{\textfraction}{0.15}
\renewcommand{\floatpagefraction}{0.8}

%\RequirePackage[section]{placeins}

%% SECTION TITLES
\RequirePackage[compact,sf,bf]{titlesec}
\titleformat{\section}[block]
  {\fontsize{15}{17}\bfseries\sffamily}
  {\thesection}
  {0.4em}{}
\titleformat{\subsection}[block]
  {\fontsize{12}{14}\bfseries\sffamily}
  {\thesubsection}
  {0.4em}{}
\titlespacing{\section}{0pt}{*5}{*1}
\titlespacing{\section}{0pt}{*2}{*0.2}


%% TITLE PAGE
\def\Date{\number\day}
\def\Month{\ifcase\month\or
 January\or February\or March\or April\or May\or June\or
 July\or August\or September\or October\or November\or December\fi}
\def\Year{\number\year}

\makeatletter
\def\wp#1{\gdef\@wp{#1}}\def\@wp{??/??}
\def\jel#1{\gdef\@jel{#1}}\def\@jel{??}
\def\showjel{{\large\textsf{\textbf{JEL classification:}}~\@jel}}
\def\nojel{\def\showjel{}}
\def\addresses#1{\gdef\@addresses{#1}}\def\@addresses{??}
\def\cover{{\sffamily\setcounter{page}{0}
        \thispagestyle{empty}%
        \vspace*{-2cm}
        \centerline{\raisebox{-1.8cm}{\includegraphics[width=5cm]{MBSportrait}}\hspace*{9cm} ISSN 1440-771X}\vspace{0.99cm}
        \begin{center}\Large
        Department of Econometrics and Business Statistics\\[.5cm]
        \scriptsize http://business.monash.edu/econometrics-and-business-statistics/research/publications
        \end{center}\vspace{2cm}
        \begin{center}
        \fbox{\parbox{14cm}{\begin{onehalfspace}\centering\Huge\vspace*{0.3cm}
                \textsf{\textbf{\expandafter{\@title}}}\vspace{1cm}\par
                \LARGE\@author\end{onehalfspace}
        }}
        \end{center}
        \vfill
                \begin{center}\Large
                \Month~\Year\\[1cm]
                Working Paper \@wp
        \end{center}}}
\def\pageone{{\sffamily
        \newpage%\blankpage
        \thispagestyle{empty}
        \vbox to 23cm{
        \raggedright\baselineskip=1.2cm
     {\fontsize{24.88}{30}\sffamily\textbf{\expandafter{\@title}}}
        \vspace{2cm}\par
        \hspace{1cm}\parbox{14cm}{\sffamily\large\@addresses}\vspace{1cm}\vfill
        \hspace{1cm}{\large\Date~\Month~\Year}\\[1cm]
        \hspace{1cm}\showjel\vss}}}
\def\blindtitle{{\sffamily
     \thispagestyle{plain}\raggedright\baselineskip=1.2cm
     {\fontsize{24.88}{30}\sffamily\textbf{\expandafter{\@title}}}\vspace{1cm}\par
        }}
\def\titlepage{{\cover\newpage\setstretch{1}\pageone\newpage\blindtitle}}

\def\blind{\def\titlepage{{\blindtitle}}\let\maketitle\blindtitle}
\def\titlepageonly{\def\titlepage{{\pageone\end{document}}}}
\def\nocover{\def\titlepage{{\newpage\setstretch{1}\pageone\newpage\blindtitle}}\let\maketitle\titlepage}
\let\maketitle\titlepage
\makeatother

%% SPACING
\RequirePackage{setspace}
\spacing{1.5}

%% LINE AND PAGE BREAKING
\sloppy
\clubpenalty = 10000
\widowpenalty = 10000
\brokenpenalty = 10000
\RequirePackage{microtype}

%% PARAGRAPH BREAKS
\setlength{\parskip}{1.4ex}
\setlength{\parindent}{0em}

%% HYPERLINKS
\RequirePackage{xcolor} % Needed for links
\definecolor{darkblue}{rgb}{0,0,.6}
\RequirePackage{url}

\makeatletter
\@ifpackageloaded{hyperref}{}{\RequirePackage{hyperref}}
\makeatother
\hypersetup{
     citecolor=0 0 0,
     breaklinks=true,
     bookmarksopen=true,
     bookmarksnumbered=true,
     linkcolor=darkblue,
     urlcolor=blue,
     citecolor=darkblue,
     colorlinks=true}

%% KEYWORDS
\newenvironment{keywords}{\par\vspace{0.5cm}\noindent{\sffamily\textbf{Keywords:}}}{\vspace{0.25cm}\par\hrule\vspace{0.5cm}\par}

%% ABSTRACT
\renewenvironment{abstract}{\begin{minipage}{\textwidth}\parskip=1.4ex\noindent
\hrule\vspace{0.1cm}\par{\sffamily\textbf{\abstractname}}\newline}
  {\end{minipage}}


\usepackage[T1]{fontenc}
\usepackage[utf8]{inputenc}

\usepackage[showonlyrefs]{mathtools}
\usepackage[no-weekday]{eukdate}

%% BIBLIOGRAPHY

\makeatletter
\@ifpackageloaded{biblatex}{}{\usepackage[style=authoryear-comp, backend=biber, natbib=true]{biblatex}}
\makeatother
\ExecuteBibliographyOptions{bibencoding=utf8,minnames=1,maxnames=3, maxbibnames=99,dashed=false,terseinits=true,giveninits=true,uniquename=false,uniquelist=false,doi=false, isbn=false,url=true,sortcites=false}

\DeclareFieldFormat{url}{\texttt{\url{#1}}}
\DeclareFieldFormat[article]{pages}{#1}
\DeclareFieldFormat[inproceedings]{pages}{\lowercase{pp.}#1}
\DeclareFieldFormat[incollection]{pages}{\lowercase{pp.}#1}
\DeclareFieldFormat[article]{volume}{\mkbibbold{#1}}
\DeclareFieldFormat[article]{number}{\mkbibparens{#1}}
\DeclareFieldFormat[article]{title}{\MakeCapital{#1}}
\DeclareFieldFormat[article]{url}{}
%\DeclareFieldFormat[book]{url}{}
%\DeclareFieldFormat[inbook]{url}{}
%\DeclareFieldFormat[incollection]{url}{}
%\DeclareFieldFormat[inproceedings]{url}{}
\DeclareFieldFormat[inproceedings]{title}{#1}
\DeclareFieldFormat{shorthandwidth}{#1}
%\DeclareFieldFormat{extrayear}{}
% No dot before number of articles
\usepackage{xpatch}
\xpatchbibmacro{volume+number+eid}{\setunit*{\adddot}}{}{}{}
% Remove In: for an article.
\renewbibmacro{in:}{%
  \ifentrytype{article}{}{%
  \printtext{\bibstring{in}\intitlepunct}}}

\AtEveryBibitem{\clearfield{month}}
\AtEveryCitekey{\clearfield{month}}

\makeatletter
\DeclareDelimFormat[cbx@textcite]{nameyeardelim}{\addspace}
\makeatother
\renewcommand*{\finalnamedelim}{%
  %\ifnumgreater{\value{liststop}}{2}{\finalandcomma}{}% there really should be no funny Oxford comma business here
  \addspace\&\space}
  

\wp{??/18}
\jel{C10,C14,C22}



\author{Thiyanga S~Talagala, Rob J~Hyndman, George~Athanasopoulos}
\addresses{\textbf{Thiyanga S Talagala}\newline
Department of Econometrics and Business Statistics, \newline Monash
University, VIC 3800, Australia.
\newline{Email: \href{mailto:thiyanga.talagala@monash.edu}{\nolinkurl{thiyanga.talagala@monash.edu}}}\newline Corresponding author\\[1cm]
\textbf{Rob J Hyndman}\newline
Department of Econometrics and Business Statistics, \newline Monash
University, VIC 3800, Australia.
\newline{Email: \href{mailto:rob.hyndman@monash.edu}{\nolinkurl{rob.hyndman@monash.edu}}}\\[1cm]
\textbf{George Athanasopoulos}\newline
Department of Econometrics and Business Statistics, \newline Monash
University, VIC 3145, Australia.
\newline{Email: \href{mailto:george.athanasopoulos@monash.edu}{\nolinkurl{george.athanasopoulos@monash.edu}}}\\[1cm]
}

\date{\sf\Date~\Month~\Year}
\makeatletter
 \lfoot{\sf Talagala, Hyndman, Athanasopoulos: \@date}
\makeatother

%% Any special functions or other packages can be loaded here.
\usepackage{algorithm}
\usepackage{algpseudocode}
\usepackage{amsthm}
\usepackage{amsmath} 
\usepackage{rotating}
\usepackage{ctable}

\def\sectionautorefname{Section}


\usepackage{amsthm}
\newtheorem{theorem}{Theorem}[section]
\newtheorem{lemma}{Lemma}[section]
\theoremstyle{definition}
\newtheorem{definition}{Definition}[section]
\newtheorem{corollary}{Corollary}[section]
\newtheorem{proposition}{Proposition}[section]
\theoremstyle{definition}
\newtheorem{example}{Example}[section]
\theoremstyle{definition}
\newtheorem{exercise}{Exercise}[section]
\theoremstyle{remark}
\newtheorem*{remark}{Remark}
\newtheorem*{solution}{Solution}
\begin{document}
\maketitle
\begin{abstract}
A crucial task in time series forecasting is the identification of the
most suitable forecasting method. We present a general framework for
forecast model selection using meta-learning. A Random Forest is used to
predict the best forecasting method using only time series features. The
proposed framework has been evaluated using time series from the M1 and
M3 competitions, and is shown to yield accurate forecasts comparable to
several benchmarks and other commonly used automated approaches of time
series forecasting. A key advantage of our algorithm is that the
time-consuming part of building the random forest can be handled in
advance of the forecasting task. So when a time series
\end{abstract}
\begin{keywords}
Time Series, Forecasting, Time Series Features, Random Forest,
Meta-learning, Algorithm selection problem
\end{keywords}

\section{Introduction}\label{intro}

Forecasting is a key activity for any business to operate efficiently.
The rapid advances in computing technologies have enabled businesses to
keep track of large number of time series variables. Hence, it is
becoming increasingly common to have to regularly forecast many millions
of time series. For example, large scale businesses may be interested in
forecasting sales, cost, and demand for their thousands of products
across different locations, warehouses, etc. Technology companies such
as Google collect many millions of daily time series such as web-click
logs, web search counts, queries, revenue, number of users for different
services, etc., and require fast and accurate automatic forecasts.
However, the scale of these tasks have raised some computational
challenges that we seek to address by proposing a new fast algorithm for
model selection and time series forecasting.

When there are a large number of time series to be forecast, there are
at least three possible forecasting strategies: (1) a single method may
be used to provide forecasts across all time series; (2) a framework can
be developed to select the most appropriate forecasting method for each
series; (3) several methods can be applied for each individual series
and the resulting forecasts combined. It is very unlikely that a single
method will consistently outperform its competitors across all time
series, so we reject strategy 1. Because our focus is on fast, scalable
forecasting, we also reject the combination approach (despite it often
being the most accurate of the three strategies), as the computational
requirements are much greater than for strategy 2. We adopt the approach
of selecting an individual forecasting method for each time series to be
forecast.

However, selecting the most appropriate model for a given time series
can also be problematic. Two of the most commonly used automatic
algorithms are the automated Exponential Smoothing Algorithm (ETS) of
\textcite{Hyndman2002} and the automated ARIMA algorithm of
\textcite{Hyndman2008}. Both algorithms are implemented in the forecast
package in R \autocite{forecast}. In this paradigm, a class of models is
selected in advance, and many models within that class are estimated for
each time series. The model with the smallest AICc value is chosen and
used to compute forecasts. This approach relies on the expert judgement
of the forecaster in first selecting the most appropriate class of
models to use, as it is not usually possible to compare AICc values
\emph{between} model classes due to differences in the way the
likelihood is computed, and the way initial conditions are handled.

An alternative approach, which avoids selecting a class of models
\emph{a priori}, is to use a simple ``hold-out'' test set; but then
there is often insufficient data to draw a reliable conclusion. To
overcome this problem, time series cross-validation can be used
\autocite{hyndman2014forecasting}; then models from many different
classes may be applied, and the model with the lowest cross-validated
MSE selected. However, this increases the computation involved
considerably (at least to order \(n^2\) where \(n\) is the number of
series to be forecast).

Clearly, there is a need for a fast, accurate algorithm to automate
forecasting model selection. We propose a general meta-learning
framework using features of the time series to select the class of
models, or even the specific model, to be used for forecasting. The
model selection process is carried out using a classification algorithm
--- we use the time series features as inputs, and the best forecasting
algorithm as the output. The classification algorithm can be built using
a large historical collection of time series, in advance of the real
forecasting exercise (so it is an ``offline'' procedure). Then, when we
have a new time series to forecast, we can quickly compute its features,
use the pre-trained classification algorithm to identify the best
forecasting model, and produce the required forecasts. Thus, the
``online'' part of our algorithm requires only feature computation, and
the application of a single forecasting model, with no need to estimate
large numbers of models within a class, or to carry out a
computationally-intensive cross-validation procedure.

The rest of this paper is organized as follows. We review the related
work in \autoref{litreview}. In \autoref{methodology} we explain the
detailed components and procedures of our proposed framework for
forecast model selection. In \autoref{Mcomp} we presents the results,
followed by the conclusions and future work in \autoref{discussion}.

\section{Literature Review}\label{litreview}

\subsection{Time series features}\label{time-series-features}

Rather than work with the time series directly in the ``instance
space'', we propose analysing time series via an associated ``feature
space''. A time series feature is any measurable characteristic of a
time series. For example, \autoref{fig:fig1} shows the instance-based
representation of six time series taken from the M3 competition
\autocite{makridakis2000m3} while \autoref{fig:fig2} shows a
feature-based representation of same time series. Here only two features
are considered: the strength of seasonality and the strength of trend,
calculated based on the measures introduced by \textcite{wang2009rule}.
Time series in the lower left quadrant of \autoref{fig:fig2} are
non-seasonal but trended, while there is only one series with both high
trend and high seasonality. We also see how the degree of seasonality
and trend varies between series. Other examples of time series features
include autocorrelation, spectral entropy and measures of
self-similarity and non-linearity.

\begin{figure}

{\centering \includegraphics[width=\textwidth]{figure/fig1-1} 

}

\caption{Instance-based representation of time series}\label{fig:fig1}
\end{figure}

\begin{figure}

{\centering \includegraphics[width=0.7\linewidth]{figure/fig2-1} 

}

\caption{Feature-based representation of time series}\label{fig:fig2}
\end{figure}

\textcite{fulcher2014highly} introduced 9000 operations to extract
features from time series. The choice of the most appropriate set of
features depends on the purpose of the study. In this paper we select
time series features for the purpose of forecast model selection. The
features chosen should have the highest discriminatory power to select a
good forecasting method, while keeping the set of features as small as
possible in order to limit the computational time.

\subsection{What makes features useful for forecasting model
identification?}\label{what-makes-features-useful-for-forecasting-model-identification}

\textcite{reid1972comparison} pointed out that the performance of
various forecasting methods changes according to the nature of data, and
if the reasons for these variations are explored they may be useful in
selecting the most appropriate model. In response to the results of the
M3-Competition \autocite{makridakis2000m3}, similar ideas have been
reported by others, who have argued that the characteristics of various
time series may provide useful insights into which forecasting methods
are most appropriate to forecast a given time series
\autocites{hyndman2001s}{lawrence2001s}{armstrong2001s}.

Many time series forecasting techniques have been developed to capture
specific feature(s) of time series that are common in a particular
discipline. For example, GARCH models have been introduced to account
for time-varying volatility in financial applications. Hence, an
appropriate set of features reveals the structure of the time series
there by uncovering the underlying best forecasting method. This is the
main advantage of using features for forecasting model selection.
Further, as discussed in the introduction, the existing approaches of
forecasting model selection involve estimating several models on all
time series and then selecting the method which provides the accurate
forecasts for the hold-out set. Feature-based model selection approaches
avoid the time associated with this trial and error procedure as we do
not need to try several models on each series.

Following the idea of feature-based representation of time series
several researchers have introduced rules for forecasting based on
features \autocites{collopy1992rule}{adya2001automatic}{wang2009rule}.
\textcite{kang2017visualising} applied principal component analysis to
project a large collection of time series into 2D feature space to
visualize what makes a particular forecasting method perform well or not
in a particular domain or subset of time series. The features they
considered are spectral entropy, first-order auto-correlation
coefficient, strength of trend, strength of seasonality, seasonal period
and optimal Box-Cox transformation parameter. On the side, they proposed
a method for generating new time series based on features.

\subsection{Algorithm Selection and Meta-learning
Approach}\label{algorithm-selection-and-meta-learning-approach}

John Rice is an early and strong proponent of the idea of meta learning
which he called algorithm selection problem(ASP)\autocite{rice1976}. The
term \emph{meta-learning} started to appear with the emerge of
machine-learning literature. The Rice's framework for algorithm
selection is shown in \autoref{fig:rice}.

\begin{figure}

{\centering \includegraphics[width=\textwidth]{figures/RiceFramework} 

}

\caption{Rice's framework for Algorithm Selection Problem (reproduced from Smith-Miles, 2009)}\label{fig:rice}
\end{figure}

There are four main components in Rice's framework for ASP. The problem
space \(P\) represents the data sets used in the study. The feature
space, \(F\) is the range of measures that characterize the problem
space \(P\). The algorithm space \(A\) is a list of suitable candidate
algorithms which we can use to find solutions to the problems in \(P\).
The performance metric \(Y\) is a measure of algorithm performance such
as accuracy, speed, etc. Rice's formal definition of algorithm selection
problem is \autocite{smith2009cross}:

\begin{definition}
\label{def2}
For a given problem instance $x \in P$, with features $f(x) \in F$, find the selection mapping $S(f(x))$ into algorithm space $A$, such that the selected algorithm $\alpha \in A$ maximizes the performance mapping $y(\alpha(x)) \in Y$.
\end{definition}

The main challenge in ASP is to identify the selection mapping \(S\)
from feature space to algorithm space. Even though, Rice's framework
articulate a conceptually rich framework, it is not clear the
implementation from feature space to the algorithm space. This gives
rise to the meta-learning approach. The main difference between the
Rice's framework and the meta-learning approach is that the
meta-learning framework consist of a offline(training) phase and online
(prediction) phase. In the offline phase, the mapping \(S\) is learned
based on a previous record of training examples. This is performed by
using a \emph{meta-learner} which can be a any supervised learning
algorithm. In other words, the meta-learner establishes rules to link
the relationship between the feature space and the algorithm space. In
the context of meta-learning, the inputs for the meta-learner are known
as \emph{meta-features} and instances in the algorithm space are the
\emph{output labels} of the meta learner. The database consists of both
input-features and output-labels is called \emph{meta-data}. In the
offline phase of the algorithm input-features are extracted from new
data and passed into the meta-learner constructed in the offline phase
to predict output-labels of new data. The framework has attracted lot of
research in recent as researchers are increasingly investigating how to
identify the most suitable method among the existing algorithms for
solving a problem rather than developing new algorithms.

\subsection{Previous Work on Forecasting Model Selection using
Meta-Learning}\label{previous-work-on-forecasting-model-selection-using-meta-learning}

To address the problem of forecasting model selection several attempts
have been taken in the context of meta-learning framework. In this
section we briefly review some work that made an advance in this area.
In general forecasting model selection problems addressed in the
literature can be framed according the definition of Rice's algorithm
selection problem as follow:

\begin{definition}
\label{def2}
For a given time series $x \in P$, with features $f(x) \in F$, find the selection mapping $S(f(x))$ into algorithm space $A$, such that the selected algorithm $\alpha \in A$ minimizes forecasting accuracy error metric $y(\alpha(x)) \in Y$ on the test set of the time series.
\end{definition}

The methods introduced in literature differ with respect to the way they
define - problem space(\(A\)), features(\(F\)), forecasting accuracy
measure(\(Y\)) and selection mapping (\(S\)).

\textcite{collopy1992rule} introduced 99 rules based on 18 features of
time series to make forecasts for economics and demographic time series.
This work was further improved by \textcite{armstrong2001s} reducing the
human intervention. \textcite{shah1997model} used simple descriptive
statistics, features related to dimentionality of time series and
autocorrelation and partial autocorrelation based features to classify
time series using discriminant analysis. More specifically, the features
are, number of observations, ratio of the number of turning points to
the length of the series, ratio of number of step changes, skewness,
kurtosis, coefficient of variation, autocorrelation at lag 1, 2, 3 and 4
and partial autocorrelation at lag 2, 3, and 4. Casting his work in
Rice's framework: \(P=203\) quarterly series of M-competition
\autocite{makridakis1982accuracy}; \(A=3\) forecasting methods, namely
simple exponential smoothing, Holt-Winters exponential smoothing with
multiplicative seasonality and basic structural time series model;
\(Y=\) mean squared error for the hold-out sample. The mapping \(S\) is
learnt by using discriminant analysis.

The work done by \textcite{prudencio2004meta} was the first to use the
term ``meta-learning'' in the context of time series model selection.
They studied the applicability of meta-learning approaches for
forecasting model selection based on two case studies. Using the
notations of Rice's algorithm, for first case study, their algorithm
space \(A\), contained simple exponential smoothing method and
time-delay neural network, \(Y=\) mean absolute error, and the mapping
\(S\) was learnt by using C4.5 decision tree algorithm. The feature
space, \(F\) consisted 14 features, namely length, autocorrelation
coefficients, coefficient of variation, skewness, kurtosis, and test of
turning points to measure the randomness of the time series. For the
second study, random walk, Holt's linear exponential smoothing and AR
models were considered in to the algorithm space \(A\). The problem
space \(P\) contained, yearly series of M3 competition
\autocite{makridakis2000m3}, F = subset of features from the first study
and Y = ranking based on error. Beyond the task of forecasting model
selection they used NOEMON approach to rank the algorithms
\autocite{kalousis1999noemon}.

\textcite{lemke2010meta} studied the applicability of different
meta-learning approaches for forecasting model selection. Their
algorithm space \(A\) contained ARIMA models, exponential smoothing
models and neural networks model. In addition to the statistical
measures such as standard deviation of detrended series, skewness,
kurtosis, length, strength of trend, Durbin-Watson statistics of
regression residuals, number of turning points, step changes,
predictability measure, non-linearity, largest Lyapunov exponent, and
auto-correlation and partial-autocorrelation, he used frequency domain
based features. The feed forward neural network, decision tree and
support vector machine approaches were considered to learn the mapping
\(S\).

\textcite{wang2009rule} used meta-learning framework to determine rules
to provide recommendations as to which forecast method to use to
generate forecasts. The rules are derived based on the relationship
between time series features and forecasting method suitability. In
order to evaluate the forecasting method suitability they introduced a
new measure, namely, \emph{simple percentage better(SPB)} which
calculate the forecasting accuracy of a method against the forecasting
accuracy error of random walk model. They used 9 features of time series
namely, strength of trend, strength of seasonality, serial correlation,
non linearity, skewness, kurtosis, self-similarity, chaos and
periodicity. Later, this set of features has become a benchmark tool for
many studies related to feature-based analysis of time series. According
to the notation of Rice, the algorithm space \(A=8\) forecasting methods
namely, exponential smoothing, ARIMA, neural networks and random walk
model, and the mapping \(S\) between the features and performance of
forecasting methods were learned by C4.5 algorithm for building decision
trees. In addition to that, to understand the nature of time series in a
two-dimensional setting they used SOM clustering on the features of the
time series. The set of features introduced by \textcite{wang2009rule}
was later used by \textcite{widodomodel} to develop a meta learning
framework for forecasting model selection. The authors further reduced
the dimensionality of time series by performing principal component
analysis on the features.

More recently, \textcite{kuck2016meta} proposed meta-learning framework
based on neural networks for forecasting model selection. Adopting the
notation of Rice, \(P = 78\) time series from NN3-competition was used
to build the meta-learner. They introduced a new set of features based
on forecasting errors. The average symmetric mean absolute percentage
error was used to identify the best forecasting method for each series.
They classify their forecasting models in the algorithm space as \(A\),
single, seasonal, seasonal-trend and trend exponential smoothing. The
mapping \(S\) was learned by using feed-forward neural network. Further,
they evaluated the performance of different set of features for
forecasting model selection.

\section{Methodology}\label{methodology}

The overview of the proposed framework is presented in
\autoref{fig:framework}. The offline and the online part of the
framework are shown in blue and red colours respectively. A
classification algorithm (meta-learner) is trained under the offline
phase and then is used to select appropriate forecasting models for new
series (online phase).

In order to train our classification algorithm, we need a large
collection of time series which are similar to those that we will be
forecasting. We assume that we have an essentially infinite population
of time series, and we take a observed sample of them in order to train
our classification algorithm. The new time series we aim to be
forecasting are thought of as additional draws from the same population.
Hence, any conclusions made from the classification framework refer only
to the population from which the sample has been selected. We may call
this the target population. It is important to have a well defined
target population to which the classification framework is applied in
advance to avoid miss-applying the classification rules generated based
on the available sample. We denote the collection of time series used
for training the classifier as the ``reference set''. We split each time
series within the reference set into a training set and a test set. From
each training set we compute a range of time series features and also
fit a selection of potential models. The calculated features form the
input vector to the classification algorithm. Using the fitted models we
generate forecasts and identify the ``best'' model for each training set
based on forecast error measure (eg: MASE) calculated over the test set.
The models deemed ``best'' form the output labels for the classification
algorithm. The pseudo code for the algorithm implemented in this paper
is given in Algorithm \autoref{alg:algo-lab}. In the following sections,
we briefly introduce components associated with the offline part of the
algorithm.

\begin{figure}

{\centering \includegraphics[width=\textwidth]{figures/framework} 

}

\caption{Proposed framework (blue: offline phase, red: online phase)}\label{fig:framework}
\end{figure}

\begin{algorithm}
  \caption{Identification of "best" forecast method for a new time series}
  \label{alg:algo-lab}
  \begin{algorithmic}[1]
    \Statex \textbf{Offline phase}
    \Statex \text{Given:}
    \Statex \hspace{1cm}$O=\{t_1, t_2, \dots,t_n\}:$ the collection of $n$ observed time series
      \Statex \hspace{1cm}$L:$ the set of class labels (eg: ARIMA, ETS, SNAIVE, etc.)
         \Statex \hspace{1cm}$F:$ the set of functions to calculate time series features
         \Statex \hspace{1cm}$nsim:$ number of series to be simulated
         \Statex \hspace{1cm}$B:$ number of trees in the random forest
         \Statex \hspace{1cm}$mtry:$ number of features to be selected at each node
     \Statex \text{Output:}
      \Statex \hspace{1cm}\text{a random forest classifier} 
      \Statex
     \Statex \textit{Prepare the reference set}
    \Statex For $i=1$ to $n$
    \State Fit ARIMA and ETS models to $t_i$
    \State Simulate $nsim$ number of time series from each model in step 2
    \State The time series in $O$ and simulated time series in step 3 create the reference set $R=\{t_1, t_2, \dots,t_n, t_{n+1},\dots,t_N\}$ where $N = n + nsim$.
    \Statex 
    \Statex \textit{Prepare the meta-data}
    \Statex For $j=1$ to $N$
    \State Split $t_i$ into training set and test set
    \State Calculate features $F$ based on the training set 
    \State Fit $L$ models to the training set
    \State Calculate forecasts for the test set from each model
    \State Calculate forecast error measure over the test set for all models in $L$
    \State Select the model with the minimum forecast error
    \State Meta-data: input features - step 7, output labels - step 11
     \Statex
    \Statex \textit{Train a random forest classifier}
    \State Train a random forest classifier based on the meta-data
    \State {Random forest: the ensemble of trees $\{T_b\}_1^B$} 
    \Statex
     \Statex \textbf{Online phase}
    \Statex \text{Given:}
    \Statex \hspace{1cm}\text{the random forest classifier in step 14} 
     \Statex \text{Output:}
      \Statex \hspace{1cm}\text{class labels for newly arrived time series $t_{new}$} 
  \State For $t_{new}$ calculate features $F$
  \State Let $\hat{C_b}(t_{new})$ be the class prediction of the $b^{th}$ random forest tree. Then class label for $t_{new}$ is $\hat{C_{rf}}(t_{new})=majorityvote{\hat{C_b}(t_{new})}$
   \end{algorithmic}
\end{algorithm}

\subsection{Augmenting the observed sample with simulated
series}\label{augmenting-the-observed-sample-with-simulated-series}

In practice, we may wish to augment our reference set by simulating new
time series that are similar to those from the population. This process
may be useful when our observed sample of time series is too small to
build a reliable classifier. Alternatively, we may wish to add more of
some types of time series to the reference set in order to get a more
balanced sample for the classification. In order to produce simulated
series that are similar to our population, we use several standard
automatic forecasting algorithms such as ETS or automated ARIMA models,
and then simulate multiple time series from the selected model within
each model class. Assuming the models used are appropriate to the data,
this ensures that the simulated series are similar to those in the
population. Note that, this is done in the off-line phase of the
algorithm, the computational time in producing these simulated series is
of no real consequence.

\subsection{Input: features}\label{input-features}

Our proposed algorithm depends crucially on finding features that enable
identification of a suitable model for the given time series. Therefore,
the features used should capture the dynamic structure of the time
series which is important for identifying models for forecasting. Such
features are measures related to the auto-correlation structure of the
time series, the trend, the seasonality (if the data is seasonal) and
nonlinearlity.

The purpose of this feature-based framework is to lessen the workload
associated with the trial and error procedure of model selection and
thereby reduce this time. Therefore, we must be conscious that the time
taken to calculate the input features should be significantly less than
the time taken to estimate parameters of all of the candidate models
required in a model selection procedure. Furthermore, interpretability,
robustness to outliars, scale and length independence are some other
factors that should be taken into consideration when selecting features
for this classification problem. A comprehensive description of the
features used in the experiment is specified in \autoref{sec:features}.

\subsection{Output: labels}\label{output-labels}

The task of our classification framework is to identify the best
forecasting method for a given time series. In this study we define the
best forecast method for a given time series as the model which performs
well for the out of sample according to some accuracy measure (for
example, MAPE, MASE etc). The measure we use to select the best
forecasting model could vary according to the purpose of forecasting and
it could be either multiple or single criteria.

In real life it is not possible to train time series among all possible
classes of time series models to identify the best forecasting method,
but at least we have to consider enough possibilities so that the
algorithm can be used for out of sample classification with high
confidence. However, the models to be considered will depend on the
specific population of time series models we need to forecast. For
example, if we have only non-seasonal time series, and no chaotic
features, we may wish to restrict our models to random walks, white
noise, ARIMA processes and ETS processes. Even in this scenario, the
number of possible models can be quite large. In order to identify the
best forecasting method for each series, all the methods considered are
run on all time series in the reference set and generate forecasts. It
is important to note that the model estimation is done on the training
part of each series and forecasts are compared with the values in the
test set. It is apparent that this step is computationally intensive and
time consuming, as all methods have to be tried on each and every series
in the reference set. Since, this is done in the offline phase the time
and the computational cost associated with this is not a problem.

\subsection{Random forest algorithm}\label{random-forest-algorithm}

Having described the data inputs and outputs for the supervised
classification approach, we briefly review the random forest algorithm
for classification \autocite{breiman2001random}. Random Forest is an
ensemble learning method that grows a large number of decision trees
using a two-step randomization process.

The algorithm works as follow: Let
\(S=\{(\underline{x_1}, y_1), (\underline{x_2}, y_2), \dots, (\underline{x_N}, y_N)\}\)
be the reference set, where the input \(\underline{x_i}\), is a
\(1\times F\) vector of features, the output, \(y_i\) corresponds to the
class label of the \(i^{th}\) observation, and \(F\) is the total number
of features. \(N\) is the number of training examples in the reference
set. Each tree in the forest is grown based on a bootstrap sample of
size \(N\) from the reference set. At each node of the tree, randomly
select \(f\) features from the full set of features \(F\). The best
split is selected among those \(f\) features. The split which results in
most homogeneous sub-node is considered as the best split. Various
measures have been introduced to evaluate the homogeneity of subnodes,
such as; classification error rate, Gini index and cross entropy
\autocite{friedman2001elements}. In this study, we use Gini index to
evaluate the homogeneity of a particular split. The trees are grown to
the largest extent possible without pruning. To determine the class
label for a new instance, features are calculated and passed down the
trees. Then each tree gives a prediction and the majority vote over all
individual trees lead to the final decision. In this work, we used the
randomForest package \autocite{liaw2002randomforest} in R which
implements the Fortran code for the Random Forest classification, by
\textcite{breiman2004random}.

\section{Application to M-competition data}\label{Mcomp}

To test how well our proposed framework can identify the suitable
forecasting models, we use the time series of the M1-competition
\autocite{makridakis1982accuracy} and M3 competition
\autocite{makridakis2000m3}. The R package Mcomp \autocite{hyndmanmcomp}
accompanies the data of M1 and M3 competitions. The proposed algorithm
is applied to yearly, quarterly and monthly series separately. We run
two experiments on each case. In the first experiment we treat the time
series of M1 competition as the \emph{observed sample} and the time
series of M3 competition as the collection of \emph{new time series}. We
run the second experiment by considering the M3 data as the
\emph{observed sample} and the M1 data as the \emph{new time series}
collection. Note that, the all phrases are consistent with the
components in \autoref{fig:framework}. This allow us to compare our
results with those of the literature. In both experiments, we fit ARIMA
and ETS models to the full length of each series in the corresponding
\emph{observed sample} based on \texttt{auto.arima} and \texttt{ets}
functions in the forecast package \autocite{Hyndman2008}. Subsequently,
from each model we further simulate 1000 series. For monthly time
series, we further simulate 100 series from each model to fasten the
offline calculation process. The lengths of simulated time series are
set equal to the lengths of the corresponding series in the
M-competition.

As shown in \autoref{fig:framework}, the task of constructing the
meta-database contains two main components, (i) identification of
\emph{output-label} and (ii) feature computation process. In the
forthcoming paragraph we will first discuss the process of identifying
\emph{output-labels}, followed by an overview of the features used in
the experiment.

The output-labels we consider in this experiment are,

\begin{enumerate}
\def\labelenumi{\roman{enumi})}
\item
  White noise process (WN)
\item
  AR/ MA/ ARMA
\item
  ARIMA
\item
  Random walk with drift (RWD)
\item
  Random walk (RW)
\item
  The Theta method
\item
  STL-AR: STL decomposition method is applied to the time series and AR
  models is fitted to the seasonally adjusted time series while seasonal
  naive method is used to forecast the seasonal component.
\item
  Exponential Smoothing Model (ETS) without trend and seasonal
  components
\item
  ETS with trend component and without seasonal component
\item
  ETS with damped trend component and without seasonal component
\end{enumerate}

In addition to the above ten(10) output labels, for seasonal data, we
further include the following five class labels,

\begin{enumerate}
\def\labelenumi{\roman{enumi})}
\setcounter{enumi}{10}
\item
  ETS with trend and seasonal components
\item
  ETS with damped trend and seasonal components
\item
  ETS with seasonal components and without trend component
\item
  SARIMA
\item
  Seasonal naive method.
\end{enumerate}

Therefore, in accordance to Rice's framework, for yearly data, the
algorithm space(\(A\)) contains 10 models while for seasonal data the
algorithm space(\(A\)) contains 15 models.

Inorder to identify the output label: ``best'' model, RW, RWD, Theta,
STL-AR, Seasonal naive (only for seasonal time series), WN are
implemented on training set of each series and forecasts are produced
for the whole of the test sets. In addition, we further use
\texttt{auto.arima} and \texttt{ets} functions in the forecast package
to identify suitable AR/MA/ARMA, ARIMA, SARIMA and ETS model. The model
model corresponds to the smallest MASE \autocite{hyndman2006another} for
the test set is selected as the \emph{output-lable}.

\subsection{Feature computation process}\label{sec:features}

We use a set of 25 features for yearly data and a set of 30 features for
seasonal data, spanning from simple attributes like, length of series,
to slightly complex ones, like spectral entropy. Some of the features
are already established features from previous studies
\autocites{wang2009rule}{hyndman2015large}{kang2017visualising}. We have
also added some new features that we believe provide some useful
information. These are summarized in \autoref{feature}. For a full
description of each feature please refer to Appendix 1.

\newcommand{\boxedcheckmark}
  {{\ooalign{$\Box$\cr\hidewidth$\checkmark$\hidewidth}}}

\begin{table}[!p]
\centering\small
\caption{Feature description}
\label{feature}
\begin{tabular}{llp{7.5cm}cc}
\toprule
\multicolumn{2}{c}{Feature} & Description & non-seasonal &  seasonal\\ 
\midrule
1  & N              & length of the time series                                                                        & $\checkmark$  & $\checkmark$ \\
2  & trend          & strength of trend                                                                                & $\checkmark$  & $\checkmark$\\
3  & seasonal       & strength of seasonality                                                                          & -             & $\checkmark$ \\
4  & linearity      & linearity                                                                                        & $\checkmark$  & $\checkmark$ \\
5  & curvature      & curvature                                                                                        & $\checkmark$  & $\checkmark$ \\
6  & spikines       & spikines                                                                                         & $\checkmark$  & $\checkmark$ \\
7  & e\_acf1        & first autocorrelation coefficient of the remainder series                                        & $\checkmark$  & $\checkmark$ \\
8  & stability      & stability                                                                                        & $\checkmark$  & $\checkmark$ \\
9  & lumpiness      & lumpiness                                                                                        & $\checkmark$  & $\checkmark$ \\
10 & entropy        & spectral entropy                                                                                 & $\checkmark$  & $\checkmark$ \\
11 & hurst          & Hurst exponent                                                                                   & $\checkmark$  & $\checkmark$ \\
12 & nonlinearity   & nonlinearity                                                                                     & $\checkmark$\ & $\checkmark$ \\
13 & alpha          & Holt's linear trend model-$\hat\alpha$                                                           & $\checkmark$  & $\checkmark$ \\
14 & beta           & Holt's linear trend model-$\hat\beta$                                                            & $\checkmark$  & $\checkmark$\\
15 & hwalpha        & Holt-Winters addtive method - $\hat\alpha$                                                       & -             & $\checkmark$ \\
16 & hwbeta         & Holt-Winters addtive method - $\hat\beta$                                                        & -             & $\checkmark$ \\
17 & hwgamma        & Holt-Winters addtive method - $\hat\gamma$                                                       & -             & $\checkmark$ \\
18 & ur\_pp         & test statistic based on Phillips-Perron test                                                     & $\checkmark$  & - \\
19 & ur\_kpss       & test statistic based on kpss test                                                                & $\checkmark$  & - \\
20 & x\_acf1        & first autocorrelation coefficient of the original series                                        & $\checkmark$  & $\checkmark$ \\
21 & diff1x\_acf1   & first autocorrelation coefficient of the differenced series                                      & $\checkmark$  & $\checkmark$ \\
22 & diff2x\_acf1   & first autocorrelation coefficient of the twiced-differenced series                               & $\checkmark$  & $\checkmark$ \\
23 & x\_acf5        & sum of squared of first 5 autocorrelation coefficients of the original series                    & $\checkmark$  & $\checkmark$ \\
24 & diff1x\_acf5   & sum of squared of first 5 autocorrelation coefficients of the differenced series                 & $\checkmark$  & $\checkmark$ \\
25 & diff2x\_acf5   & sum of squared of first 5 autocorrelation coefficients of the twice-differenced series           & $\checkmark$  & $\checkmark$ \\
26 & seas\_acf1     & autocorrelation coefficient at first lag                                                         & -             & $\checkmark$ \\
27 & sediff\_acf1   & first autocorrelation coefficient of the seasonally-differenced series                           & -             & $\checkmark$\\
28 & sediff\_seacf1 & first autocorrelation coefficient at the first seasoanl lag of the seasonally-differenced series & -             & $\checkmark$ \\
29 & sediff\_acf5   & sum of squared of first 5 autocorrelation coefficients of the seasonally-differenced series      & -             & $\checkmark$ \\
30 & lmres\_acf1    & first autocorrelation coefficient of the residual series of linear trend model                   & $\checkmark$  & - \\
31 & x\_pacf5       & sum of squared of first 5 partial autocorrelation coefficients of the original series            & $\checkmark$  & $\checkmark$ \\
32 & diff1x\_pacf5  & sum of squared of first 5 partial autocorrelation coefficients of the differenced series         & $\checkmark$  & $\checkmark$ \\
33 & diff2x\_pacf5  & sum of squared of first 5 partial autocorrelation coefficients of the twice-differenced series   & $\checkmark$  & $\checkmark$ \\
\bottomrule
 \end{tabular}
\end{table}

\subsection{Model calibration}\label{model-calibration}

Our reference set is imbalanced: some classes contains significantly
more cases than the other classes. The degree of class imbalance to some
extent by augmenting the observed sample with simulated time series. The
random forests algorithm is highly sensitive to the class imbalance
\autocite{breiman2001random}. We use three approaches to address the
class imbalance in the data: i) incorporate class priors into the random
forest classifier, and ii) use Balanced Random Forest(BRF) algorithm
introduced by \textcite{chen2004using} and iii) re-balancing the
reference set with down sampling. Note, that BRF algorithm is different
from down-sampling approach. In down sampling thereference set is
pre-processed by down-sampling the majority class into the size of the
minority class which is potentially discard some useful information. We
compare the results of above three random forests to the random forest
classifier build on imbalanced data. The RF algorithms are implemented
by the randomForest R package \autocite{liaw2002randomforest}. The class
priors are introduced through the option \texttt{classwt}. We use
reciprocal of class size as class priors. In each case the two
parameters of the of RF algorithm are set as: number of
trees(\emph{ntree}) - 1000, and number of randomly selected
features(\emph{mtry}) - one third of the total number of features. The
number of trees are limited to 1000 to fasten the online calculation
process. The Random forest trained on unbalanced data (RF-unbalanced)
and

\subsection{Summary of the main results}\label{sec:results}

The matrices of Pearson correlation coefficients for all the features in
the reference sets of each experiments are presented in
\autoref{fig:cormat}. Although the correlations among particular
features are of interest the focal point of \autoref{fig:cormat} is the
entire matrix of correlation coefficients. The degree of variability in
the Pearson's correlation coefficients between features indicate the
diversity of the selected features. In other words the features we used
were able to capture the different characteristics of the time series.
Further, the structure of correlation matrices are similar to each other
Random forest with class priors (RF-class priors) outperform the other
methods.

\begin{figure}

{\centering \includegraphics[width=\textwidth]{figure/cormat-1} 

}

\caption{ Correlation matrix plots (A-Experiment1 yearly, B- Experiment 2 yearly, C- Experiment 1, quarterly, D- Experiment 2 quarterly, E - Experiment 1 monthly, F - Experiment 2 monthly)}\label{fig:cormat}
\end{figure}

We now presents the results of our experiments on yearly, quarterly and
monthly series separately. We build separate random forest classifiers
to yearly data, quarterly data and monthly data. In each case, for the
second experiment(M3-observed sample, M1-new series) we take a subset of
simulated time series to train the RF-unbalanced and RF-class priors as
\texttt{randomForest} package does to facilitate in handling large data
sets. The subsets are selected randomly according to the proportions of
output-labels in the observed samples. This ensures that our reference
set shares the similar characteristics of the observed sample. The
principal component analysis is use to visualize the relationship
between feature-space of the different time series collections: observed
time series, simulated time series, subset of simulated time series and
new time series. For each experiment, principal component analysis(PCA)
is performed on all the features in the observed sample. Then we project
the simulated time series and the new time series into the PCA space of
the observed data. The results are shown in \autoref{fig:pca1} -
\autoref{fig:pca3}, for yearly, quarterly and monthly data respectively.
On each experiment the first three principal components are plotted
against each other. The point on each graph represents a time series.

The accuracy of our method is compared against following benchmarks and
other commonly used approaches of forecasting:

\begin{enumerate}
\def\labelenumi{\arabic{enumi}.}
\tightlist
\item
  automated ARIMA algorithm of \textcite{Hyndman2008}
\item
  automated ETS algorithm of \textcite{Hyndman2008}
\item
  Random walk with drift (RWD)
\item
  Random walk model (RW)
\item
  White noise process (WN)
\item
  Theta method
\item
  STL-AR method
\item
  seasonal naive (for seasonal data)
\end{enumerate}

The automated ARIMA and ETS algorithms are implemented using
\texttt{auto.arima} and \texttt{ets} functions available in the forecast
package in R\autocite{Hyndman2008}. Each method is implemented on the
training set and forecasts are computed up to the full length of the
test set. Then we compute the MASE for each forecast horizon, by
averaging the MASE across all series in the the collection of new times
series. Further, to assist in the evaluation of the proposed framework,
for each forecast horizon we rank our method compared to the other
methods listed above and and average ranking over all forecast horizons
are computed. The results are given in \autoref{YM3} -- \autoref{M1M}.
The MASE value corresponds to the best performing method in each
category is highlighted in bold.

\textbf{Yearly data}

For the yearly series in M1 competition, the first 3 principal
components explain 62.47\% of the variation of features. For the yearly
series in M3 competition, the first three principal components explains
62.19\% of the total variance. As seen in \autoref{fig:pca1}, simulated
time series are able to fill the gap appeared between the points in the
observed sample. By augmenting the reference set with simulated time
series we were able to increase the diversity and evenness(to some
extent) of the feature space of observed time series. Further, in both
experiments, all the \emph{observed time series} falls within the space
of all simulated data. This guarantees that we have not lost any feature
structures of the observed sample. The remaining plots in
\autoref{fig:pca2}-\autoref{fig:pca3} can be interpreted similar to
\autoref{fig:pca1}.

The \autoref{YM3} to \autoref{YM1} compare the performance of our
proposed framework to the benchmark methods. For each method we
calculate out-of-sample MASE over the forecast horizons 1-h, and average
over all time series. For yearly series of M3 competition random walk
with drift model seem to be inferior to the other methods. The average
MASE values corresponds to the RF-class priors are slightly higher than
the results of random walk with drift. For yearly series of M1
competition RF-unbalanced and RF-class priors consistently forecast more
accurately than random walk with drift model.

\begin{figure}

{\centering \includegraphics[width=\textwidth]{figure/pca1-1} 

}

\caption{Distribution of yearly time series in the PCA space; results of experiment 1 (observed sample-M1, new time series - M3) are shown in panels 1.A- 1.C and results of experiment 2 (observed sample-M3, new time series - M1) are shown in panels 2.A- 2.C, on each graph colour scheme is green-simulated time serie, yellow-subset of simulated time series, black-observed time series, orange-new time series}\label{fig:pca1}
\end{figure}

\begin{table}[!h]
\centering
\caption{Experiment 1 (Observed sample - M1): Forecast accuracy measures for 645 M3-yearly series}
\label{YM3}
\begin{tabular}{lrrrrrrr}
\hline
 & \multicolumn{6}{c}{Average of forecasting horizons: 1-h} \\ \hline
 & 1  & 1-2  & 1-3  & 1-4 & 1-5 & 1-6 & Average Rank \\ \hline
 RF-unbalanced & 1.06  &  1.42 & 1.83  & 2.20 & 2.54 & 2.85 & 3.50 \\ 
 RF-class priors& \bf{1.03}  & 1.37  & 1.78  & 2.14 & 2.47 & 2.77 & 1.83 \\ 
auto.arima &  1.11 & 1.48  & 1.90  & 2.28 & 2.63 & 2.96 & 6.83 \\ 
ets&   1.09  & 1.44  & 1.84 & 2.20 & 2.54 & 2.86 & 4.17\\ 
WN&    6.54  & 6.91  & 7.22 & 7.48 &7.76  & 8.07 & 9.00\\ 
RW &    1.24  & 1.68  & 2.11 & 2.48 & 2.83 & 3.17 &8.00 \\ 
RWD&   \bf{1.03}  & \bf{1.36}  & \bf{1.74} & \bf{2.05} & \bf{2.35} & \bf{2.63} &1.17 \\ 
STL-AR & 1.09  &  1.47 &  1.89 & 2.27 & 2.62 & 2.95 & 5.50 \\ 
Theta & 1.12  & 1.47  & 1.86  & 2.18 & 2.48 & 2.77 & 4.17 \\ \hline
\end{tabular}
\end{table}

\begin{table}[!h]
\centering
\caption{Experiment 2 (Observed sample - M3): Forecast accuracy measures for 181 M1-yearly series}
\label{YM1}
\begin{tabular}{lrrrrrrr}
\hline
 & \multicolumn{6}{c}{Average of forecasting horizons: 1-h} \\ \hline
 & 1  & 1-2  & 1-3  & 1-4 & 1-5 & 1-6 & Average Rank \\ \hline
 RF-unbalanced & \bf{0.97}  & \bf{1.39}  & 1.93  & 2.42 & 2.90 & 3.37 & 1.67 \\ 
 RF-class priors& 1.02  & 1.40 &  \bf{1.92} & \bf{2.40} & \bf{2.87}  & \bf{3.33} &1.33  \\ 
auto.arima &  1.06 & 1.47  & 2.01  & 2.51 & 3.00 & 3.47 & 3.50 \\ 
ets&   1.12  & 1.59  & 2.17 & 2.72 & 3.26 &3.77  & 6.00\\ 
WN&    6.38  & 7.08  & 7.92 & 8.59 & 9.28 & 10.01 & 9.00\\ 
RW &    1.35  & 2.00  & 2.80 & 3.50 & 4.19 & 4.89 & 8.00\\ 
RWD&    1.03  & 1.44  & 2.00 & 2.51 & 3.01 & 3.49 & 3.33 \\ 
STL-AR & 1.10 &  1.51 &  2.07 & 2.55 & 3.04 & 3.52 & 5.00 \\ 
Theta & 1.15  & 1.70  & 2.38  & 3.00 & 3.59 & 4.19 & 7.00 \\ \hline
\end{tabular}
\end{table}

\subsubsection{Quarterly data}\label{quarterly-data}

The first 3 principal components quarterly time series in the M1
competition explain 62.40\% of the total variance of the features while
for the quarterly series in the M3-competition data, the amount of
variation explained by the first 3 principal components is 64.75\%.

\autoref{M3Q} to \autoref{M1Q} summarize the results for quarterly data.
The results of RF-class priors outperform the the benchmark methods.
However, the average MASE of Theta method for 1-18 slightly lower than
the RF-unbalanced and RF-class priors.

\begin{figure}

{\centering \includegraphics[width=\textwidth]{figure/pca2-1} 

}

\caption{Distribution of quarterly time series in the PCA space; results of experiment 1 (observed sample-M1, new time series - M3) are shown in panels 1.A- 1.C and results of experiment 2 (observed sample-M3, new time series - M1) are shown in panels 2.A-2.C, on each graph colour scheme is green-simulated time serie, yellow-subset of simulated time series, black-observed time series, orange-new time series}\label{fig:pca2}
\end{figure}

\begin{table}[!h]
\centering
\caption{Experiment 1 (Observed sample - M1): Forecast accuracy measures for 756 M3 - quarterly series}
\label{M3Q}
\begin{tabular}{lrrrrrrrrrl}
\hline
 & \multicolumn{8}{c|}{Average of forecasting horizons: 1-h}    &  \\ \hline
 &  1& 1-2 & 1-3 & 1-4 & 1-5 & 1-6 & 1-7 & 1-8  & Average rank  \\ \hline
RF-unbalanced& 0.58 & \bf{0.65} & \bf{0.73} & \bf{0.81} & \bf{0.88} & \bf{0.96} & \bf{1.04} & 1.12 & 1.25  \\ 
RF-class priors & 0.58&  0.66& 0.74 & 0.81 & 0.89 & 0.97 & 1.05 & 1.13   & 2.38   \\ 
auto.arima & 0.58 & 0.66  & 0.75 & 0.85 & 0.93 & 1.01 & 1.10 & 1.19 & 4.25 \\ 
ets & \bf{0.56} & \bf{0.65}  & \bf{0.73} & 0.82 & 0.91 & 0.99 & 1.08 & 1.17 & 2.75   \\ 
WN & 3.25 & 3.35 & 3.46 & 3.59 & 3.63 & 3.70 & 3.78 & 3.87 & 10.00   \\ 
RW & 1.14 & 1.12 & 1.17 & 1.16 & 1.25 & 1.32 & 1.41 & 1.46 &  7.38  \\ 
RWD & 1.20 & 1.18 & 1.23 & 1.17 & 1.29 & 1.36 & 1.44 & 1.47 & 8.38   \\ 
STL-AR & 0.70 & 0.90 & 1.08 & 1.27 & 1.44 & 1.60 & 1.75 &  1.91 & 7.88   \\ 
Theta & 0.62 & 0.68 & 0.76 & 0.83 & 0.90 & 0.97 & 1.04 &  \bf{1.11} & 3.25  \\ 
Snaive & 1.11 & 1.10 & 1.08 & 1.09 & 1.21 & 1.30 & 1.36 & 1.43 &  6.25  \\ \hline
\end{tabular}
\end{table}

\begin{table}[!h]
\centering
\caption{Experiment 2 (Observed sample - M3): Forecast accuracy measures for 203 M1 - quarterly series}
\label{M1Q}
\begin{tabular}{lrrrrrrrrrl}
\hline
 & \multicolumn{8}{c}{Average of forecasting horizons: 1-h}    &  \\ \hline
 &  1& 1-2 & 1-3 & 1-4 & 1-5 & 1-6 & 1-7 & 1-8  & Average rank  \\ \hline
 RF-unbalanced& \bf{0.77} & \bf{0.85} & \bf{0.95} & \bf{1.08} & \bf{1.22} & \bf{1.36} & \bf{1.48} & \bf{1.59} & 1.00  \\ 
RF-class priors & 0.79 &0.88  & 0.99 & 1.12 & 1.28 & 1.41 &  1.53& 1.65 &2.25    \\ 
auto.arima & 0.85 & 0.94 & 1.05 & 1.19 & 1.37 & 1.53& 1.67 & 1.80 & 5.00  \\ 
ets & 0.78 & 0.89  & 0.98 & 1.11 & 1.28 & 1.42 & 1.54 & 1.66 &  2.50  \\
WN & 3.97 & 4.14 & 4.16 & 4.27 & 4.35 & 4.45 & 4.52 & 4.64 & 10.00   \\ 
RW & 0.97 & 1.10 & 1.25 & 1.35 & 1.52 & 1.67 & 1.83 & 1.95 & 7.13  \\ 
RWD & 0.95 & 1.04 & 1.19 & 1.26 & 1.42 & 1.56 & 1.71 & 1.81 &  6.00  \\ 
STL-AR & 0.96 & 1.20 & 1.41 & 1.63 & 1.85 & 2.05 & 2.23 &  2.43 & 8.50   \\ 
Theta & 0.79 & 0.90 & 1.00 & 1.13& 1.29 & 1.42 & 1.55 &  1.67 &  3.75 \\ 
Snaive & 1.52 & 1.53 & 1.53 & 1.56 & 1.74 & 1.86 & 1.98 & 2.08 & 8.38   \\ \hline
\end{tabular}
\end{table}

\subsubsection{Monthly data}\label{monthly-data}

For monthly series of M1 competition the first three principal
components capture 78.07\% of the variability in the 30 features, while
for the monthly series of M3 competition the amount of variation
captured by the first three principal components is 65.97\%. According
to the results of \autoref{M3M} and \autoref{M1M}, RF-unbalanced and
RF-class priors seem to be inferior to other methods for long-term
forecast horizons (h=1 to 18).

\begin{figure}

{\centering \includegraphics[width=\textwidth]{figure/pca3-1} 

}

\caption{Distribution of monthly time series in the PCA space; results of experiment 1 (observed sample-M1, new time series - M3) are shown in panels 1.A- 1.C and results of experiment 2 (observed sample-M3, new time series - M1) are shown in panels 2.A- 2.C, on each graph colour scheme is green-simulated time serie, black-observed time series, orange-new time series}\label{fig:pca3}
\end{figure}

\begin{table}[!h]
\centering
\caption{Experiment 1 (Observed sample - M1): Forecast accuracy measures for 1428 M3 - monthly series}
\label{M3M}
\begin{tabular}{lrrrrrrrr}
\hline
 & \multicolumn{6}{c}{Average of forecasting horizons: 1-h}    &  \\ \hline
 &  1-4& 1-6 & 1-8 & 1-10 & 1-12 & 1-18  & Average rank  \\ \hline
RF-unbalanced&  0.66& 0.69 & 0.72 & 0.75 & 0.75 & 0.78 & 5.17  \\ 
RF-class priors &  0.65& 0.68 & 0.71 & 0.74 & \bf{0.74} & \bf{0.77}  & 4.00  \\ auto.arima & 0.61 & 0.65 & 0.69 & 0.72 & 0.75 & 0.88 &  2.67   \\  
ets & \bf{0.59}  & \bf{0.64} & \bf{0.68} & \bf{0.72} & \bf{0.74} & 0.86 & 1.67   \\  
WN & 2.06 & 2.08 & 2.10 & 2.13 & 2.15 & 2.27 & 12.00   \\  
RW &0.91  &0.97  & 1.01 & 1.04 & 1.04 & 1.17 & 10.33  \\ 
RWD &0.90  & 0.96 &1.00  &1.03  & 1.02 & 1.14 & 9.17  \\ 
STL-AR & 0.73 & 0.81 & 0.90 & 0.98 & 1.04 & 1.27 & 8.83  \\ 
Theta & 0.63 & 0.67 & 0.72 & 0.75 & 0.77 & 0.89 & 5.67    \\ 
Snaive& 0.95 & 0.97 & 0.97 & 0.98 & 0.98 &  1.14& 9.00  \\ \hline
\end{tabular}
\end{table}

\begin{table}[!h]
\centering
\caption{Experiment 2 (Observed sample - M3): Forecast accuracy measures for 617 M1 - monthly series}
\label{M1M}
\begin{tabular}{lrrrrrrrl}
\hline
 & \multicolumn{6}{c}{Average of forecasting horizons: 1-h}    &  \\ \hline
 &  1-4& 1-6 & 1-8 & 1-10 & 1-12 & 1-18  & Average rank  \\ \hline
RF-unbalanced&  0.72& 0.78 & 0.83 & 0.88 &\bf{ 0.89} & \bf{0.97} & 2.50  \\ 
RF-class priors &  0.71& 0.78 & 0.83 & 0.89 & 0.91 & 0.99  & 2.83  \\ 
auto.arima & 0.73 &0.81  & 0.87 & 0.94 & 0.99 & 1.16 &  6.83   \\  
ets &0.68  & 0.76 & 0.82 & 0.88 & 0.93 & 1.07 & 2.50   \\  
WN & 2.06 & 2.09 & 2.12 & 2.14 & 2.18 & 2.28 & 12.00   \\  
RW &1.18  &1.24 & 1.31 & 1.34 & 1.33 & 1.47 &  10.00 \\ 
RWD &1.19  & 1.27 &1.37  &1.40  & 1.39 & 1.55 & 11.00  \\ 
STL-AR & 0.79 & 0.91 & 0.99 & 1.09 & 1.17 & 1.39 & 8.33  \\ 
Theta & \bf{0.68} & \bf{0.75} & \bf{0.81} & \bf{0.87} & 0.91 & 1.04 & 1.67    \\ 
Snaive& 1.09 & 1.11 & 1.11 & 1.13 & 1.14 &  1.31&   8.67\\ \hline
\end{tabular}
\end{table}

\section{Discussion and Conclusions}\label{discussion}

In this paper we propose a novel framework for forecasting model
selection using mete-learning approach. Our proposed framework oriented
towards the automatic selection of forecasting methods based on time
series features. The basis of our algorithm is to use the knowledge of
past performance of different forecasting methods on different time
series to identify the best forecasting method for a new series.The
major contributions of this work are following,

First, we proposed a framework for forecast model identification. Our
proposed framework is not problem specific and can be applied to any
large collection of time series.

Second, we introduce a simple set of time series features that are
useful in identifying ``best'' forecast method of a given time series.

Third, in contrast to the existing approaches we have proposed a new
method to create meta-data base by simulating new time series that are
similar to those from the population.

Finally, we have used a new set of class labels to train the classifier.
When evaluating the ``best'' forecast method for a given time series,
there could be several candidates for a given time series which satisfy
the criterium of evaluating the ``best'' method. In such circumstances,
it does not matter which model is going to be selected as long as they
provide the same forecast.

Our proposed framework shown to yield accurate forecasts comparable to
several benchmarks and other commonly used approaches of forecasting.
The main advantage of our method is parameters need not to be estimated
on several models to identify the best forecasting method.

Note that we have not made a comparison of time with the benchmark
methods. However, for real-time forecasting, our framework involves only
the calculation of features and to make the prediction based on the
random forest classifier. These steps do not involve substantial
computation and can be easily parallisable to fasten for a given
computing budget. For future work, we will explore the use of other
classification algorithms and test for several large scale real time
series data sets.

\newpage

\section*{Appendix}\label{appendix}
\addcontentsline{toc}{section}{Appendix}

\textbf{Length of time series}

The length of time series is the number of observations that constitute
it. The appropriate forecasting methods depend largely on how many
observations are available. For example, shorter series tend to provide
better forecasts with more simple models such as random walk, naive
method. On the other hand, for long time series (say up to 200), models
with time-varying parameters gives best forecast as it helps to capture
the inner structural changes of the model. In this experiment we do not
consider the models with time-varying parameter to our algorithm space
as we do not have such long time series. However, we include this as a
feature as the length of the series vary relatively large.

\textbf{STL-decomposition based features: strength of trend, strength of
seasonality, linearity, curvature, spikiness and first autocorrelation
coefficient of the remainder series}

The features strength of trend, strength of seasonality, linearity,
curvature, spikiness and first autocorrelation coefficient of the
remainder series are calculated based on the STL-decomposition of the
time series. In the following description, our notations are as follows:
We represent a time series \(Y\) of length \(N\) as
\(y_1, y_2, \dots,y_N\). First, the Box-Cox transformation is applied to
the time series. The reasons for applying Box-Cox transformation: i) to
stabilize the variance, ii) to make the seasonal effect additive, and
iii) to make the data normally distributed. The transformed series is
denoted by \(Y_{t}^*\). The basic decomposition structure of the time
series is denoted by: \(Y_t^*=T_t+S_t+E_t\), where \(T_t\) denotes the
trend in time series, \(S_t\) denotes the seasonal component, while
\(E_t\) is the remainder component \autocite{cleveland1990stl}. Further,
the detrended series \(X_t\) is \(X_t=Y_t^*-T_t\), the deseasonalized
series is to be define as \(Z_t=Y_t^*-S_t\), and the remainder series,
\(R_t\), is defined as \(R_t=Y_t^*-T_t-S_t\).

\textbf{Strength of trend}

The long-term increase or decrease in time series data is called the
trend \autocite{hyndman2014forecasting}. The strength of trend is
measured by comparing the variance of de-trended series and the original
series as follow \autocite{wang2009rule}: \[
    Trend =1- \frac{var(R_{t})}{var(Z_{t})}.
\] The values of this feature range between 0 and 1.

\textbf{Strength of seasonality}

The seasonality pattern occurs when a time series shows a pattern of
repetitive behaviour over a year within a fixed period. The strength of
seasonality is computed as follows\autocite{wang2009rule}: \[
    Seasonality =1- \frac{var(R_{t})}{var(X_{t})}.
\]The values of this feature range between 0 and 1.

\textbf{Linearity and Curvature}

The features linearity and the curvature are computed based on the
coefficients of a quadratic regression of the form

\[T_t=\beta_0+\beta_1time_t+\beta_2time_t^2+\epsilon_t\]

where, \(time=1, 2, \dots,N\). The estimated value of \(\beta_1\) is
used as a measure of linearity while the estimated value of \(\beta_2\)
is considered as a measure of curvature. The features have been used by
\textcite{hyndman2015large}.

\textbf{Spikiness}

The feature spikiness occurs when the time series is affected by sudden
drops or rise. \textcite{hyndman2015large} introduced an index to
measure spikiness as follow:
\[spikiness=var\left(\frac{var(R_t)\times N-1-d}{N-2}\right);\] where
\(d=(R_t-mean(R_t))^2\). Note that \(R_t\) is the remainder component
calculated based on STL-decomposition.

\textbf{First autocorrelation coefficient of the remainder series}

We compute the first autocorrelation coefficient of the remainder
series. The first autocorrelation coefficient calculated based on the
remainder series does not influence by seasonality and trend present in
the series.

\textbf{Stability and lumpiness}

A time series is stable if it has a constant mean and a constant
variance over time. The features ``stability'' and ``lumpiness'' are
calculated based on tiled windows (windows cannot be overlapped on top
of each other). For each window, mean and the variance are calculated.
The feature stability is calculated based on the variance of means while
the lumpiness is the variance of variances.

\textbf{Spectral entropy of a time series}

Spectral entropy of a time series is an information theory based measure
which can be used as an measure of forecastability of a time series. We
use the measure introduced by \textcite{goerg2013forecastable} to
estimate the spectral entropy. It estimates the Shannon entropy of the
spectral density of a univariate (or multivariate) normalized spectral
density. The spectral density of a univariate time series \(y_t\) can be
defined as, \[f_y(\lambda)=\frac{S_y(\lambda)}{\sigma^2_y},\] where
\(S_y(\lambda)\) represents spectrum of a univariate stationary process
which is the Fourier transformation of the autocovariance function,
\[S_y(\lambda)=\frac{1}{2\pi}\sum_{j=-\infty}^{\infty}\gamma_{y}(j)e^{ij\lambda},\]
and \(\lambda \in [-\pi, \pi]\), and \(i=\sqrt{-1}\).

The Shanon entropy of \(f_y(\lambda)\) is define as,
\[ H_{s,a}(y_t):=-\int_{-\pi}^{\pi}f_y(\lambda)log_af_y({\lambda})d\lambda,\]
where \(a>0\) is the logarithm base. Since the periodogram is not a
consistent estimator for \(S_y(\lambda)\), weighted overlapping segment
averaging(WOSA) introduced by \textcite{nuttall1982spectral} was used to
estimate \(S_y(\lambda)\).

The R package ForeCA (Forecastable Component Analysis) available at CRAN
accompanies this work \autocite{Foreca}. As the name suggests ForeCA
introduces a dimension reduction technique for time series analysis
using the frequency domain properties of time series to determine the
forecastability. This measure is calculated on the original series.
Series that are easy to forecast should have a small value for the
measure.

\textbf{The Hurst exponent}

The Hurst exponent is used to measure long-term memory of time series.
We use the method presented in \textcite{wang2009rule} to estimate the
Hurst exponent. The Hurst exponent is estimated using the relation
\(H=d+0.5\), where d, is fractal dimension of FARIMA(0, d, 0).
Parameters are estimated using the maximum likelihood estimators. The
likelihood is approximated using the method illustrated by
\textcite{haslett1989space}. To fit FARIMA models we the fradiff package
available in CRAN \autocite{fracdiff} which accompanies the work of
\textcite{haslett1989space}.

\textbf{Nonlinearity}

To measure the degree of nonlinearity of the time series, we use
Teräsvirts's neural network test for nonlinearity as in
\textcite{wang2009rule}.

\textbf{Parameter estimates of Holt's linear trend model}

The forecasting equations and two-smoothing equations in Holt's linear
trend model can be expressed as follow:

\begin{table}[!h]
\centering
\begin{tabular}{rl}
Forecast equation: &  $\hat{y}_{t+h|t}=l_t + hb_t$ \\
 Level equation: & $l_t = \alpha y_t + (1-\alpha)(l_{t-1}+b_{t-1})$ \\
Trend equation: &  $b_t=\beta^*(l_t-l_{t-1})+(1-\beta^*)b_{t-1}$
\end{tabular}
\end{table}

where \(\alpha\) is the smoothing parameter for the level, and
\(\beta^*\) is the smoothing parameter for the trend. These parameters
can vary between 0 and 1. The notations are as in
\textcite{hyndman2014forecasting}. We include the parameter estimates of
both \(\alpha\) and \(\beta\) to our feature set.

\textbf{Parameter estimates of Holt-Winters additive method}

The forecasting equations and three component equations for Holt-Winters
additive method is:

\begin{table}[!h]
\centering
\begin{tabular}{rl}
Forecast equation: &  $\hat{y}_{t+h|t}=l_t + hb_t+s_{t-m+h_m^+}$ \\
 Level equation: & $l_t = \alpha (y_t-s_{t-m}) + (1-\alpha)(l_{t-1}+b_{t-1})$ \\
Trend equation: &  $b_t=\beta^*(l_t-l_{t-1})+(1-\beta^*)b_{t-1}$\\
Seasonal equation: &  $s_t=\gamma(y_t+l_{t-1}-b_{t-1})+(1-\gamma)s_{t-m}.$
\end{tabular}
\end{table}

For mathematical background and notations, we refer the reader to
\textcite{hyndman2014forecasting}. We use the parameter estimates of
\(\alpha\), \(\beta\) and \(\gamma\) to our feature set in case of
seasonal time series.

\textbf{Unit root test statistics based on Phillips-Perron test}

The test regression for Phillips-Perron test is,
\[ y_t=\alpha+(\phi -1)y_{t-1}+ \epsilon_t. \]

The hypotheses of interest are,

\begin{table}[!h]
\centering
\begin{tabular}{rl}
$H_0: \phi = 1$ vs
$H_1: |\phi| < 1$. 
\end{tabular}
\end{table}

The test statistic we use as a feature is,
\[Z = T(\hat{\phi}-1)-\frac{1}{2}\frac{T^2SE(\hat{\pi})}{\hat{\sigma^2}}(\hat{\lambda}^2-\hat{\sigma^2}).\]
The terms \(\hat{\sigma}^2\) and \(\hat{\lambda}^2\) are consistent
estimates of the variance parameters,
\[\sigma^2=\lim_{T\to\infty}T^{-1}\sum_{t=1}^{T}E[\epsilon_t^2],\]
\[\lambda^2=\lim_{T\to\infty}\sum_{t=1}^{T}E[T^{-1}S_T^2],\] where
\(S_T = \sum_{t=1}^{T}\epsilon_t\). The sample variance of the least
squares residual \(\hat{\epsilon}_t\) is a consistent estimate of
\(\sigma^2\), and the Newey-West long-run variance estimate of
\(\epsilon_t\) using \(\hat{\epsilon}_t\) is a consistent estimate of
\(\lambda^2\).

\textbf{Unit root test statistics based on KPSS test}

The test regression is, \[y_t=c+\delta t+\phi y_{t-1}+\epsilon_t.\] The
hypotheses of interest are,

\begin{table}[!h]
\centering
\begin{tabular}{rl}
$H_0: \phi = 1$ vs
$H_1: |\phi| < 1$. 
\end{tabular}
\end{table}

The test statistic we use as a feature is,
\[Z=\big( T^{-2}\sum_{t=1}^{T}\hat{S}_t^2\big)/\hat{\lambda}^2\] where
\(\hat{S}_t=\sum_{j=1}^t\hat{\epsilon}_j\), \(\hat{\epsilon}_t\) is the
least squares residuals and \(\hat{\lambda}^2\) is a consistent estimate
of the long-run variance of \(\epsilon_t\) using \(\hat{\epsilon_t}\).

Unit root tests based features are calculated using the functionality in
package \texttt{urca}\autocite{pfaff2016package}.

\textbf{Autocorrelation coefficient based features}

The autocorrelation coefficients measure the strength of the linear
relationship between lagged values of a time series. We calculate
first-order autocorrelation coefficient and sum of squares of first five
autocorrelation coefficients of the original series, first-difference
series and second-difference series and seasonal differenced series (for
seasonal data). These autocorrelation based are useful in identifying,
i) stationary vs non-stationary processes, ii) random vs non-random
processes, iii) difference stationary processes and seasonality present
in the series.

\textbf{First-order autocorrelation coefficient of the residual of
linear trend model}

A linear regression model is fitted considering
\(Y = \{y_1, y_2, \dots, y_n\}\) as the dependent variable and and time
\(1, 2,\dots, n\) as the independent variable. Then the first-order
autocorrelation coefficient of the residual series is calculated. The
purpose of including this feature is to discriminate between trend
stationary and difference-stationary processes. If \(Y\) is
trend-stationary and if a deterministic trend is fitted then the
residuals are white noise. On the other hand if the \(Y\) is
difference-stationary and a deterministic trend is fitted, residuals
follow a random walk model.

\textbf{Partial-autocorrelation based features}

Partial-autocorrelation measures the relationship between \(y_t\) and
\(y_{t-k}\) after removing the effects of other time lags --
\(1, 2, 3, \dots, k-1\). We calculate the sum of squares of first five
partial autocorrelation coefficients of the original series,
first-difference series and second-difference series. Hence, this gives
three features to our experiment. Partial-autocorrelation coefficients
play an important role in Box-Jenkins\autocite{box2015time} approach to
time series modelling as it helps to determine number of AR terms to be
included in both AR(P) and ARIMA(P, d, Q).

\newpage

\printbibliography[title=References]

\end{document}
